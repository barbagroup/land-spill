\documentclass[letterpaper]{article}
\usepackage[utf8]{inputenc}
\usepackage[margin=1in]{geometry}
\usepackage{graphicx}
\usepackage[backend=biber, 
            style=ieee, 
            citestyle=numeric-comp,
            url=false,
            isbn=false,
            giveninits=true,
            hyperref=auto,
            eprint=false,
            maxcitenames=2,
            mincitenames=1]{biblatex}
\usepackage{hyperref}
\usepackage{parskip}
\usepackage{enumitem}
\usepackage{fontspec}

\setlength{\parindent}{0pt}
\setlength{\parskip}{12pt}
\renewcommand{\baselinestretch}{1.4}
\setmainfont{Noto Serif}
\addbibresource{references.bib}

\begin{document}

   \noindent           
   {\bf \Large Note for overland flow modeling}\\
   {\bf \normalsize Pi-Yueh Chuang, Jul. 2018}

The start point of this note is to understand the common prediction 
models for the overland spill of non-aqueous phase liquids (NAPLs). 
NAPLs include, for example, crude oil, gasoline, petroleum products, etc. 
The spreading behavior after liquids spilled on land depends on many factors, 
such as drainage, infiltration, evaporation, weathers (e.g., rainfall \& wind), 
etc.\ \cite{Simmons2003}.
Many studies currently found, however, only consider simplified models that only
involve three factors: evaporation, infiltration, and surface flow
\ \cite{Hussein2002, Ronnie2004, Zuczek2008}. 
Adhesion of oil to land is considered by\ \cite{Farrar2005}. 
At the current stage, I focus only on the modeling of surface flow on land and 
leave other factors for the future.

I could only find a handful of research in overland oil flow. 
Some of them are more mathematical and physical 
\ \cite{Hussein2002, Simmons2003, Guo2006} 
but are limited to simple geometries, such as a horizontal plate or an incline 
with a constant gradient. 
Others are instead designed for real-world situations, complicated topographies,
and more computer-oriented
\ \cite{Ronnie2004, Farrar2005, Zuczek2008, Gin2012, Su2017}.
The underlying flow models used by some of the later ones, however, are somehow 
far away from the first principle, in my opinion.
The work of \textcite{Su2017} may be the most relevant to what we are interested 
in. 
See the section of the shallow-water equation models.

\section{Overland spill models for simple geometries}

\subsection{\textcite{Hussein2002}}

\textcite{Hussein2002} propose what may be the first model that couples surface
flow, evaporation, and infiltration. 
The flow model used is based on gravity current models\ \cite{Huppert1982, Lister1992}. 
Gravity current models study the behavior of a fluid horizontally flowing into 
another fluid with a different density. 
Examples are cold air flowing into hot air, and salt water flowing into
the fresh water. 
In the context of overland oil flow, the idea is to treat oil as a dense fluid 
flowing into a light fluid, which is air in this case.
\textcite{Hussein2002} couple the gravity current models of
\ \cite{Huppert1982, Lister1992} with an infiltration model\ \cite{Weaver1994} 
and an evaporation model\ \cite{Stiver1989}.

The gravity current models used are limited to cases of a point source or a line 
source of fluid flowing on a horizontal flat plate or an incline with a constant 
gradient in one-dimensional flow approximation.
The equations can thus provide time-dependent data of thickness profiles and
the locations of flow front. 
In the work of \textcite{Hussein2002}, however, they do not utilize the full 
gravity current models.
They only use the part that predicts the location of flow front with respect
to time. 
Thickness in their work is an averaged thickness calculated by total volume
and spill size.

The way \textcite{Hussein2002} couple the flow with infiltration and evaporation
is through the calculation of effective fluid volumes. The equation to predict
the location of the front depends on the volume of fluid. So utilizing an
effective volume, which is the net value of inflow (point/line source) and
outflow (infiltration and evaporation), couples the three factors.

\subsection{\textcite{Simmons2003}}

\textcite{Simmons2003}, on the other hand, re-derive the equation for the 
prediction of the flow front based on the work of \textcite{Hussein2002}. 
The difference lies in that the equation derived by \textcite{Simmons2003} 
depends on the averaged thickness and present front location, instead of the
flow volume.

\subsection{Other comments}

In the derivation of one of the gravity current models\ \cite{Huppert1982}, an
assumption is made before the derivation: the two fluids do not have 
significantly different viscosities. 
In overland oil spills, however, the viscosity of oil may be very different from 
air, which violates the assumption.
This will require more digging in the literature in the future if there's any 
chance we are going to use gravity current models.

Also, there are some other models in this category require further study to 
have better understandings:

\begin{enumerate}[noitemsep]
    \item \textcite{Acton2001} develop a more complicated gravity current model
        for flow on permeable plates.
    \item \textcite{Guo2006} presents a model for asphalt oil flowing on  
        sloping ground.
\end{enumerate}


\section{Overland spill models for complex topographies}

I further divided the models in this category into two sub-categories: 
the steepest route models and shallow-water equations.

\subsection{The steepest route models}

From my feeling, this type of models seems to be popular among the commercial
software for a risk-evaluation purpose. 
Due to the confidential nature of commercial software, however, papers of this 
type usually don't provide details of their models, algorithms, and 
implementations. 
Nevertheless, they are still good enough to have a basic understanding.

In the model of the steepest route, the surface flow can generally be separated 
into two processes: flow trajectory calculation and flow property calculation. 
During the flow trajectory calculation, the gradient of a cell is used to 
determine which in the 8 surrounding cells the oil in the current cell will go
toward.
Then we can obtain the path of the oil flow, beginning from the point 
source and ending at the intersection with river/lake/ocean or some other 
barriers blocking the oil flow. 
The path of the oil flow, however, does not have any other flow information, 
such as thickness distribution, time history of the front, etc.
So the flow property calculation process is carried out after finding the path.
The difference between models or software lies in the methods for calculating 
the flow property. See the following subsections for comments.

\subsubsection{\textcite{Ronnie2004}}

The flow property calculation is done right after the flow front is moved from 
one cell to the next cell by the steepest route. 
After the movement of the flow front, the flow properties at the current cell 
(the cell where the front is currently in) is calculated with the LSMS 
model\ \cite{Linden1998}. 
Data from the LSMS model are not calculated on the fly. 
A table of data is calculated in advance based on the soil property in the 
area of interest. 
During the simulation, then, flow properties are obtained from the table and 
interpolation.

The LSMS model requires a further reading. At a glance, LSMS seems to solve 
one-dimensional shallow-water equation coupled with infiltration and evaporation
models. LSMS can provide data like depth, velocity, temperature, spread and 
vaporization rates\ \cite{Ronnie2004}.

The paper gives an example of the simulation of a real case occurred at 
Bellingham, Washington on June 10th 1999.

\subsubsection{\textcite{Farrar2005}}

This paper focuses on coupling GIS technology and oil spill modeling but also 
briefly mentions the overland flow model they are using. 
The steepest route model for flow path is described in the paper, but at what 
timing during the procedure they calculate the flow property is unclear to me.

Their model includes adhesion and evaporation. 
See the paper for the adhesion and evaporation models they are using.
The velocity of the flow front is calculated by Manning's equation.

\subsubsection{\textcite{Zuczek2008}}

The work of \textcite{Zuczek2008} presents a complicated model for risk analysis
and estimated costs for response and environmental damage. The code is developed
with VBA as an extension to ArcGIS.

In a nutshell, they apply the work of \textcite{Hussein2002} to calculate flow 
properties after the path of flow is determined.

\subsection{Shallow-water equations (SWEs)}

Shallow-water equations (SWEs), also called Saint-Venant equations, is widely 
used for simulations of hydrologic problems, e.g., tsunami\ \cite{Arcos2015}, 
flood\ \cite{Bates2000, Zhang2007, Hunter2007}, 
rivers streams\ \cite{Gouta2002, Burguete2004}, 
rainfall runoff\ \cite{Esteves2000}, 
dam breaks\ \cite{Xanthopoulos1976, Brufau2000}, etc. 
In the 90s', 1D shallow-water equation is used due to limited computing power. 
However, 2D shallow-water equation becomes popular in the 21st century because of 
the advancement of computing power.

Besides the full SWEs, two types of simplified SWEs exist and are also popular:
kinematic wave approximation and diffusive wave approximation.
Kinematic wave approximation ignores the local acceleration (time derivatives), 
convection and pressure forces in the momentum equations.
This implies the gravity is in balance with the friction of fluid and ground.
On the other hand, diffusive wave approximation ignores only the local 
acceleration and convection and retains pressure, gravity, and friction effects.
One key reason for using the simplified models is to ease the need for computing 
resource.
Another reason is the concerns of numerical issues arising from solving the
full SWEs, such as numerical stability and convergence. 
See\ \cite{HEC-11993, Kazezyilmaz-Alhan2012} for more information and 
limitations of the simplified models.

Currently, I found only one possible paper regarding applying the full SWEs to 
simulate overland oil flow.
\textcite{Gin2012} present a case study of spill analysis on a 
segment of crude oil pipeline near the Edmonton South Terminal in the area of 
Sherwood Park, AB, Canada. 
The paper only briefly mentions that they are using their proprietary 
software, FLO-2D, for the overland oil flow. 
From the limited information revealed in the paper, it looks like FLO-2D is 
solving the full dynamic wave equation, which to my understanding, is another
name of the full SWEs.
No further information or details about the overland flow solver presents in the
paper. 

On the other hand, \textcite{Su2017} present a work that solves the diffusive
wave approximation of SWEs for overland crude oil flow.
They carry out simulations of the spreading of crude oil after some tanks break 
at a storage site. 
The underlying flow solver comes from the work of \textcite{Bates2000} for flood 
simulations.
\textcite{Su2017} modify the flow solver so that the friction model is based
on the Darcy-Weisbach equation, instead of Manning's equation.


\section{Other}

Here are some papers that I think may be related but I don't have time to dig
deeper:

\begin{enumerate}[noitemsep]
    \item \textcite{Paige2002} show an interface of their software that looks
        like is capable of overland flow simulations of oil. However, no useful
        information about the model presents.
    \item \textcite{Linden1998} describe a model for predicting the spreading
        and vaporization of spills of cryogenic liquids, such as liquefied 
        natural gas. The model seems to be based on the SWEs.
    \item \textcite{Ponchaut2011} utilize the SWEs to study the characteristics
        of the evaporation during spills of liquefied natural gas.
\end{enumerate}


\printbibliography
\end{document}
